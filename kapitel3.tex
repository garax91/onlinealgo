\chapter{Das $k$-Server-Problem}


\section{Einführende Bemerkungen}

\subsection{Der Greedy-Algorithmus}

Der Greedy-Algorithmus (bewege immer den Server, der am nähesten ist) ist nicht kompetitiv. Es reichen hierzu 2 Server und 3 Anfragepunkte ($a$,$b$,$c$) auf der Linie, wobei der Abstand zwischen $a$ und $b$ kleiner als der Abstand $bc$ bzw $ac$ sein muss. Die Eingabe $\sigma = (c,(a,b)^l)$ erzeugt für den Greedy-Algorithmus mit $l$ wachsende Kosten, wobei die optimalen Kosten konstant sind.

\subsection{Die $k$-Server-Vermutung}

\textbf{Vermutung 3.1} ($k$-Server-Vermutung). Für jeden metrischen Raum gibt es einen deterministischen $k$-kompetitiven Online-Algorithmus für das $k$-Server-Problem.

\textit{Vermutlich nicht wirklich relevant.} Bis heute wurde gezeigt, dass es für keinen metrischen Raum mit mindestens $k+1$ Punkten einen deterministischen Online-Algorithmus gibt, der $r$-kompetitiv für ein $r < k$ ist. Es ist aber bekannt, dass es einen deterministischen Online-Algorithmus gibt, der $(2k-1)$-kompetitiv ist.

Die $k$-Server-Vermutung konnte bisher noch nicht bewiesen oder widerlegt werden.

\textbf{Vermutung 3.2} (Randomisierte $k$-Server-Vermutung). Für jeden metrischen Raum gibt es einen randomisierten $O(\log k)$-kompetitiven Online-Algorithmus für das $k$-Server-Problem.

Bisher konnte diese Vermutung noch nicht bewiesen werden. Daher ist sie wohl nicht sonderlich relevant.

\subsection{Optimale Offline-Algorithmen}


\section{Untere Schranke für deterministische Algorithmen}

\textbf{Theorem 3.3.} Es sei $\mathcal{M} = (M,d)$ ein beliebiger metrischer Raum mit $|M| \geq k+1$. Dann gibt es für kein $r < k$ einen $r$-kompetitiven Online-Algorithmus für das $k$-Server-Problem auf $\mathcal{M}$.

\textit{Beweis.} Sei $B \subseteq M$ eine beliebige Teilmenge von $M$ mit $k+1$ Elementen und $A$ ein beliebiger fauler deterministischer Algorithmus. Da es einen Punkt mehr als Server gibt, ist immer genau ein Punkt nicht von einem Server besetzt. Deshalb wählen wir die Sequenz so, dass die Anfrage immer genau dort auftritt, wo kein Server steht.


\textbf{Lemma 3.4.} Es gilt $$\omega_A(\sigma) \geq \sum_{i=1}^{n-1}d(\sigma_i,\sigma_{i+1})$$

\textit{Beweis.} Wir haben $\sigma$ so gewählt, dass der Algorithmus $A$ bei jeder anfrage einen Server verschieben muss. Da sich der Server, der Anfrage $\sigma_i$ bearbeitet, zuvor bei $\sigma_{i+1}$ befunden hat (bzw. da wir $\sigma_{i+1}$ so gewählt haben), entstehen bei diesem Vorgang Kosten in Höhe von $d(\sigma_{i+1},\sigma_i)$. Deshalb können die Kosten insgesamt durch $$\omega_A(\sigma) \geq \sum_{i=1}^{n-1}d(\sigma_{i+1},\sigma_i) = \sum_{i=1}^{n-1}d(\sigma_i,\sigma_{i+1})$$ nach unten abschätzen.


\textbf{Lemma 3.5.} Es gilt $$\textrm{OPT}(\sigma) \leq \frac{1}{k}\sum_{i=1}^{n-1}d(\sigma_i,\sigma_{i+1})$$

\textit{Beweis.} Fehlt noch!


\section{Das $k$-Server-Problem auf Linien und Bäumen}

\subsection{Analyse des DC-Algorithmus auf der Linie}

\subsection{Analyse des DC-Algorithmus auf der Bäumen}

\subsection{Anwendungen des DC-Algorithmus}


\section{Das 2-Server-Problem im euklidischen Raum}
