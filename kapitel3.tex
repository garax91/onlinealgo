\chapter{Das $k$-Server-Problem}


\section{Einführende Bemerkungen}

\subsection{Der Greedy-Algorithmus}

Der Greedy-Algorithmus (bewege immer den Server, der am nähesten ist) ist nicht kompetitiv. Es reichen hierzu 2 Server und 3 Anfragepunkte ($a$,$b$,$c$) auf der Linie, wobei der Abstand zwischen $a$ und $b$ kleiner als der Abstand $bc$ bzw $ac$ sein muss. Die Eingabe $\sigma = (c,(a,b)^l)$ erzeugt für den Greedy-Algorithmus mit $l$ wachsende Kosten, wobei die optimalen Kosten konstant sind.

\subsection{Die $k$-Server-Vermutung}

\textbf{Vermutung 3.1} ($k$-Server-Vermutung). Für jeden metrischen Raum gibt es einen deterministischen $k$-kompetitiven Online-Algorithmus für das $k$-Server-Problem.

\textit{Vermutlich nicht wirklich relevant.} Bis heute wurde gezeigt, dass es für keinen metrischen Raum mit mindestens $k+1$ Punkten einen deterministischen Online-Algorithmus gibt, der $r$-kompetitiv für ein $r < k$ ist. Es ist aber bekannt, dass es einen deterministischen Online-Algorithmus gibt, der $(2k-1)$-kompetitiv ist.

Die $k$-Server-Vermutung konnte bisher noch nicht bewiesen oder widerlegt werden.

\textbf{Vermutung 3.2} (Randomisierte $k$-Server-Vermutung). Für jeden metrischen Raum gibt es einen randomisierten $O(\log k)$-kompetitiven Online-Algorithmus für das $k$-Server-Problem.

Bisher konnte diese Vermutung noch nicht bewiesen werden. Daher ist sie wohl nicht sonderlich relevant.

\subsection{Optimale Offline-Algorithmen}


\section{Untere Schranke für deterministische Algorithmen}

Um die untere Schranke für deterministische Algorithmen zu zeigen, müssen wir die Kosten eines beliebigen deterministischen Algorithmus $A$ nach unten, und die optimalen Kosten nach oben abschätzen.

\textbf{Theorem 3.3.} Es sei $\mathcal{M} = (M,d)$ ein beliebiger metrischer Raum mit $|M| \geq k+1$. Dann gibt es für kein $r < k$ einen $r$-kompetitiven Online-Algorithmus für das $k$-Server-Problem auf $\mathcal{M}$.

\textit{Beweis.} Sei $B \subseteq M$ eine beliebige Teilmenge von $M$ mit $k+1$ Elementen und $A$ ein beliebiger fauler deterministischer Algorithmus. Da es einen Punkt mehr als Server gibt, ist immer genau ein Punkt nicht von einem Server besetzt. Deshalb wählen wir die Sequenz so, dass die Anfrage immer genau dort auftritt, wo kein Server steht.


\textbf{Lemma 3.4.} Es gilt $$\omega_A(\sigma) \geq \sum_{i=1}^{n-1}d(\sigma_i,\sigma_{i+1})$$

\textit{Beweis.} Wir haben $\sigma$ so gewählt, dass der Algorithmus $A$ bei jeder anfrage einen Server verschieben muss. Da sich der Server, der Anfrage $\sigma_i$ bearbeitet, zuvor bei $\sigma_{i+1}$ befunden hat (bzw. da wir $\sigma_{i+1}$ so gewählt haben), entstehen bei diesem Vorgang Kosten in Höhe von $d(\sigma_{i+1},\sigma_i)$. Deshalb können die Kosten insgesamt durch $$\omega_A(\sigma) \geq \sum_{i=1}^{n-1}d(\sigma_{i+1},\sigma_i) = \sum_{i=1}^{n-1}d(\sigma_i,\sigma_{i+1})$$ nach unten abschätzen.


\textbf{Lemma 3.5.} Es gilt $$\textrm{OPT}(\sigma) \leq \frac{1}{k}\sum_{i=1}^{n-1}d(\sigma_i,\sigma_{i+1})$$

\textit{Beweis.} Wir definieren eine Klasse $C$ von Algorithmen auf einer Menge $S \subseteq B$ mit $b_1 \in S$ und $|S|=k$. Somit erhalten wir genau $k$ Algorithmen. Bei einer Anfrage $\sigma_i$, auf der kein Server steht, wird der Server von $\sigma_{i-1}$ dort hin bewegt. Dies kann bei einer Anfrage aber immer nur höchstens für einen Algorithmus der Klasse $C$ passieren. Somit entstehen durchschnittliche Kosten von $$\frac{1}{k}\sum_{i=1}^{n-1}d(\sigma_i,\sigma_{i+1})$$. Somit wissen wir, dass einer der Algorithmen aus Klasse $C$ geringere Kosten haben muss, da nicht alle Kosten größer als deren Durchschnitt sein können.


\section{Das $k$-Server-Problem auf Linien und Bäumen}

\subsection{Analyse des DC-Algorithmus auf der Linie}

\textbf{Theorem 3.6.} Der DC-Algorithmus ist $k$-kompetitiv für das $k$-Server-Problem auf der Linie.

\textit{Beweis.} Der Beweis kann nur mit Hilfe der Potentialfunktion $\Phi$ geführt werden, da der DC-Algorithmus in manchen Schritten mehr als $k$-mal so viele Kosten wie OPT hat. Die Potentialfunktion beschreibt hierbei die „Unterschiedlichkeit“ der Server-Positionen von DC und OPT.

\textbf{Lemma 3.7.} Es gilt $$\Phi'_{i-1} \leq \Phi_{i-1} + k \cdot \textrm{OPT}_i(\sigma)$$

\textit{Beweis.} Fehlt noch!

\textbf{Lemma 3.8.} Es gilt $$\Phi_{i} \leq \Phi'_{i-1} - \textrm{DC}_i(\sigma)$$

\textit{Beweis.} Fehlt noch!


\subsection{Analyse des DC-Algorithmus auf der Bäumen}

\textbf{Theorem 3.9.} Der DC-Algorithmus ist $k$-kompetitiv für das $k$-Server-Problem auf jeder Baummetrik $\mathcal{M} = (V,d)$.

\textit{Beweis.} Fehlt noch!

\subsection{Anwendungen des DC-Algorithmus}

\textbf{Korollar 3.10.} Der DC-Algorithmus ist $(N-1)k$-kompetitiv für das $k$-Server-Problem auf beliebigen Metriken mit $N$ Punkten.

\textbf{Korollar 3.11.} Der DC-Algorithmus ist $k$-kompetitiv für das Paging-Problem und das gewichtete Paging-Problem.

\section{Das 2-Server-Problem im euklidischen Raum}

Wir definieren uns die Funktion $\textrm{slack}(x,y,r) = d(x,y)+d(x,r)-d(y,r)$, die den „Durchhang“ bzw Größe des Umwegs eines Weges angibt. Aufgrund der Dreiecksungleichung ist diese nie negativ.

\textbf{Theorem 3.12.} Der Algorithmus $\textrm{SC}_{1/2}$ ist 3-kompetitiv für das 2-Server-Problem im euklidischen Einheitsquadrat.

\textit{Beweis.} Fehlt noch!
