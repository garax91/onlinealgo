\chapter{Approximation von Metriken}

\textbf{Definition 4.1.} Es sein $\mathcal{M} = (M, d)$ eine beliebige Metrik. Wir sagen, dass eine Metrik $\mathcal{M'} = (M', d')$ mit $M \subseteq M'$ die Metrik $\mathcal{M}$ dominiert, wenn $d(x, y) \le d'(x, y)$ für alle $x,y \in M$ gilt. Es sei $\mathcal{S}$ eine Menge von Metriken, die $\mathcal{M}$ dominieren, und sei $\mathcal{D}$ eine Wahrscheinlichkeitsverteiulung auf der Menge $\mathcal{S}$. Wir sagen, dass $(\mathcal{S}, \mathcal{D})$ eine $\alpha$-Approximation der Metrik $\mathcal{M}$ ist, wenn für alle $x, y \in M$ gilt:
$$E_{(M', d')\sim D} [d'(x, y)] \le \alpha \cdot d(x, y)$$
Das bedeutet, für jedes feste Paar $x$ und $y$ von Punkten aus $M$ darf die erwartete Entfernung in einer zufällig gemäß der Verteilung $\mathcal{D}$ gewählten Metrik höchstens $\alpha \cdot d(x, y)$ betragen.

\section{Approximation durch Baummetriken}

\textbf{Theorem 4.2.} Für jede Metrik $\mathcal{M}$ mit $N$ Punkten gibt es eine Menge $\mathcal{S}$ von Baummetriken, die $\mathcal{M}$ dominieren, und eine Wahrscheinlichkeitsverteilung $\mathcal{D}$ auf $\mathcal{S}$, sodass $(\mathcal{S}, \mathcal{D})$ eine $O(\log N)$-Approximation von $\mathcal{M}$ ist. Es gibt ferner einen effizienten Algorithmus, der zu jeder Metrik $\mathcal{M}$ eine solche Einbettung ($\mathcal{S}$, $\mathcal{D}$) berechnet.

\textit{Beweis.} Der Beweis hierzu wurde im Skript nicht aufgeführt, es wird auf ein Paper verwiesen. Dieses Theorem folgt aber aus Lemma 4.6, welches später bewiesen wird.

\subsection{Hierarchische Partitionen und Baummetriken}

\textbf{Definition 4.3.} Eine Partition der Metrik $\mathcal{M} = (V, d)$ mit Radius $r \ge 1$ ist eine Partition der Menge $V$ in Klassen $V_{1}, ..., V_{l}$, sodass für jede Menge $V_{i}$ ein Zentrum $c_{i} \in V$ existiert, für das $d(c_{i}, v) \le r$ für alle Punkte $v \in V_{i}$ gilt.

\textbf{Definition 4.4.} Eine hierachische Partition der Metrik $\mathcal{M} = (V, d)$ ist eine Sequenz $D_{0}, D_{1}, ... D_{\delta}$ von $\delta + 1$ Partitionen von $V$ mit den folgenden Eigenschaften:
\begin{enumerate}
\item Es gilt $D_{\delta} = {V}$, d.h. $D_{\delta}$ ist die triviale Partition der Metrik $\mathcal{M}$ mit Radius $2^{\delta}$, in der alle Punkte aus $V$ zur selben Klasse gehören.
\item Für jedes $i \le \delta$ ist $D_{i}$ eine Partition der Metrik $\mathcal{M}$ mit Radius $2^{i}$, die die Partition $D_{i+1}$ verfeinert. Das heißt, jede Klasse von $D_{i}$ ist Teilmenge einer Klasse von $D_{i+1}$.
\end{enumerate}

\textbf{Lemma 4.5.} Für jede hierachische Partition einer Metrik $\mathcal{M} = (V, d)$ dominiert die resultierende Baummetrik $d_{T}$ die Metrik $d$.

\textit{Beweis:} !!!!!TODO!!!!!

\subsection{Erzeugung von hierarchischen Partitionen}

\textit{Algorithmus:} HierPart($\mathcal{M}$ = (V, d)) erzeugt mit hilfe von Partition($\mathcal{M}$, D, $\alpha$, $\pi$) eine hierarchische Partition.

\subsection{Analyse des Streckungsfaktors}

\textit{Theorem 4.2} folgt aus folgendem Lemma:

\textbf{Lemma 4.6.} Es sei $d_{T}$ die Baummetrik, die sich aus der hierachischen Partition ergibt, die der Algorihmus HierPart($\mathcal{M}$ = (V, D)) ausgibt. Für jedes Paar x $\in$ V und y $\in$ V gilt:\\
$E[d_{T}(x, y)] \le 64 H_{N} \cdot d(x, y)$, \\
wobei $H_{N}$ die N-te harmonische Zahl bezeichnet.

\textit{Beweis:} !!!!!TODO!!!!!

\textbf{Lemma 4.7.} Für jeden Punkt $v_{l} \in V$ und jede Ebene j $\in {0, 1, ... \delta -1}$ gilt: $$Pr[A(v_{l}, j)] \le \tfrac{d(x, y)}{l \cdot 2^{j-1}}$$

\textit{Beweis:} !!!!!TODO!!!!!

\textbf{Lemma 4.8.} Für jeden Knoten $v_{l}$ gibt es maximal vier Ebenen j $\in {0, 1, ... \delta - 1}$, für die das Ereignis A($v_{l}$, j) eintreten kann.

\textit{Beweis:} !!!!!TODO!!!!!

\textbf{Beobachtung 4.9.} Für jede baummetrik $\mathcal{M}_{T} = (V_{T}, d_{T})$ aus der Menge $\mathcal{S}$ aus Theorem 4.2. gilt: \\
$\max_{x, y \in V_{T}} d_{T}(x, y) \le 8 \cdot \max_{x, y \in V} d(x, y)$.

\textit{Beweis:} !!!!!TODO!!!!!

\section{Anwendung auf das $k$-Server-Problem}


\textbf{Theorem 4.10.} Es gibt einen randomisierten Online-Algorithmus für das k-Server-Problem, der für jede Metrik mit $N$ Punkten $=(k \cdot \log N)$-kompetitiv ist.

\textit{Beweis:} !!!!!TODO
