\chapter{Scheduling}

Die Ausführungszeit von Maschiene i $\in$ M in Schedule $\pi$ ist: $L_{i}(\pi) = \tfrac{\sum_{j \in J: \pi(j)=i} p_{j}}{s_{i}}$. (Summe der Jobgrößen auf i geteilt durch Geschwindigkeit von i) \\
Der Makespan $C(\pi) = \max_{i \in M} L_{i}(\pi)$ ist die Ausführungszeit der Maschiene, die am längsten benötigt. (Dies soll minimiert werden)

\section{Identische Maschinen}

\textit{Least-Loaded-Algorithmus} weißt der Maschiene den Job zu, die momentan die kleinste Ausführungszeit hat.

\textbf{Theorem 6.1.} Der Least-Loaded-Algorithmus ist für Online-Scheduling mit identlischen Maschinen strikt $(2-\frac{1}{m})$-kompetitiv.

\textit{Beweis.} Fehlt noch!


\textbf{Theorem 6.2.} Der Longest-Processing-Time-Algorithmus ist ein $\frac{4}{3}$-Approximationsalgorithmus für Scheduling mit identischen Maschinen.

\textit{Beweis.} Fehlt noch!



\section{Maschinen mit Geschwindigkeiten}

\textbf{Lemma 6.3.} Es sei $\alpha \in \mathbb{R}_{>0}$ beliebig und $\sigma$ sei eine beliebige Eingabe für Online-Scheduling mit $\textrm{OPT}(\sigma) \leq \alpha$. Dann gibt der Algorithmus $\textrm{SLOWFIT}(\alpha)$ auf der Eingabe $\alpha$ keine Fehlermeldung aus und berechnet einen Schedule $\pi$ mit $C(\pi) \leq 2 \alpha$.

\textit{Beweis.} Fehlt noch!


\textbf{Theorem 6.4.} Der Algorithmus SLOWFIT ist ein strikt 8-kompetitiver Algorithmus für Online-Scheduling.

\textit{Beweis.} Fehlt noch!
