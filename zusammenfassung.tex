\documentclass[a4paper]{article}

\usepackage{geometry}
\geometry{margin=3cm,top=2cm,bottom=3cm}

\usepackage[ngerman]{babel}
\usepackage[utf8]{inputenc}
\usepackage{amsmath}
\usepackage{amssymb}
\usepackage{graphicx}

\renewcommand{\labelenumi}{(\alph{enumi})}

\setlength{\parskip}{\baselineskip}
\setlength{\parindent}{0pt}

\title{Klausurvorbereitung\\Online-Algorithmen}

\author{Jens Harder, Daniel Rauber}

\date{}

\begin{document}
\maketitle

\textbf{Definition 2.8} Ein randomisierter Online-Algorithmus $A$ für ein Minimierungsproblem $\Pi$ erreicht einen kompetitiven Faktor von $r \ge 1$, wenn es eine Konstante $r \in \mathbb{R}$ gibt, sodass
$$E[w_A(\sigma)] \leq r \cdot \textrm{OPT}(\sigma) + \tau$$
für alle INstanzen $\sigma \in I_\Pi$ gilt. Gilt diese Ungleichung sogar für $\tau = 0$, so ist $A$ strikt $r$-kompetitiv.

Der Unterschied zu Definition 1.1 besteht lediglich darin, dass wir hier den Erwartungswert der Kosten nutzen, da diese vom Zufall abhängen.

\end{document}
