\documentclass[a4paper]{article}

\usepackage{geometry}
\geometry{margin=3cm,top=2cm,bottom=3cm}

\usepackage[ngerman]{babel}
\usepackage[utf8]{inputenc}
\usepackage{amsmath}
\usepackage{amssymb}
\usepackage{graphicx}
\usepackage{stmaryrd}
\usepackage{mathtools}

\renewcommand{\labelenumi}{(\alph{enumi})}

\setlength{\parskip}{\baselineskip}
\setlength{\parindent}{0pt}

\title{Klausurvorbereitung\\Online-Algorithmen}

\author{Jens Harder, Daniel Rauber}

\date{}

\begin{document}
\maketitle

\section*{Definition 1.1 Kompetitiver Faktor}

\section*{Markierungsalgorithmen:}
$\sigma$: ($\overbrace{1, 2, 4, 2, 1,}^{Phase 1}$$\overbrace{3, 5, 2, 3, 5,}^{Phase 2}$$\overbrace{1, 2, 3,}^{Phase 3}$$\overbrace{4}^{Phase 4}$)

\section*{Theorem 2.1: LRU und FWF sind Markierungsalgorithmen}
Beweis: Widerspruchsbeweis: In der Phase müsste auf k + 1 verschiedene Seiten zugegriffen werden, damit eine markierte Seite verdrängt wird. $\lightning$

\section*{Theorem 2.2: Jeder Markierungsalgorithmus ist strikt k - kompetitiv}
Beweis:\\
1. Fall: es gibt nur eine Phase: keine Seite wird jemals verdrängt: Markierungsalgorithmus ist optimal \\
2. Fall: es gibt mind. 2 Phasen: Kosten $\le$ l $\cdot$ k (Anzahl Phasen mal Anzahl versch. Seiten / Phase). \\
Annahme: opt. Offline Algo. macht mind. (k + l - 2) Seitenfehler (k in der ersten Phase, in jeder weiteren Phase (außer in der letzten) mind. einen Seitenfehler). \\
\\
Teilsequenzen (k=3):\\
$\sigma$: ($\overbrace{1, 2, 4, 2, 1,}^{Phase 1}$$\overbrace{3, 5, 2, 3, 5,}^{Phase 2}$$\overbrace{1, 2, 3,}^{Phase 3}$$\overbrace{4}^{Phase 4}$) \\
$\sigma$: (1, 2, 4, 2, 1, $\overbracket{3}^{x}$, $\underbracket{5, 2, 3, 5, 1}_{T1}$, $\underbracket{2, 3, 4}_{T2}$)

pro Teilsequenz mind. 1 Seitenfehler, da z.B. in T1 k Seiten sind, die sich von x unterscheiden $\to$ 1 Seitenfehler.
Also: $\omega_{A}$($\sigma$) $\le$ l $\cdot$ k $\le$ (k + l - 2) $\cdot$ k $\le$ OPT($\sigma$) $\cdot$ k

\section*{korollar 2.3: LRU und FWF sind strikt k - kompetitiv}

\section*{Theorem 2.4: LFU und LIFO sind nicht kompetitiv}
Beweis: Gebe Gegenbeispiel an:\\
$\sigma$ = ( $1^l$, $1^l$, ..., $(k-1)^l$, ($(k, k+1)^l$). \\
$\Rightarrow$ k - 1 + 2(l - 1) Seitenfehler \\
$\omega_{LFU}$($\sigma$) = $\omega_{LIFO}$($\sigma$) $>$ r $\cdot$ OPT($\sigma$ + $\tau$ \\
$\Leftrightarrow$ k - 1 + 2 (l - 1) $>$ r $\cdot$ (k + 1) + $\tau$ für hinreichend große l

\section*{Theorem 2.5: Es gibt für kein r $<$ k einen deterministischen r - kompetitiven online Algorithmus für das Paging - Problem}
Beweis: Konstruiere Eingabe für bel. online Algo A: $\sigma_{1}$, $\sigma_{2}$, ..., $\sigma_{k}$: k verschiedene Seitem. Also hat sowohl A als auch opt. offline Algo k Seitenfehler.\\
Für $\sigma_{k + 1}$, ..., $\sigma_{k + l}$ gilt: Es kommt immer die Seite, die A vorher aus dem Cach geschmissen hat. Also macht A hier k + l Seitenfehler; LFD hat nur $\omega_{LFD}$($\sigma$) $\le$ k + 1 + $\tfrac{l}{k}$ Seitenfehler. \\
Da LFD mind. so viele Fehler macht wie OPT gilt: $\omega_{A}$($\sigma$) = k + l $>$ r $\cdot$ ( k + 1 + $\tfrac{l}{k}$) + $\tau$ $\ge$ r $\cdot$ $\omega_{LFO}$($\sigma$) + $\tau$ \\
 $\Rightarrow$  $\omega_{A}$($\sigma$) $>$ r $\cdot$ OPT($\sigma$) $\cot$ r

\section*{Deterministische Algorithmen für das Paging - Problem:}

 \begin{tabular}{| l | l |}
 \hline
  LRU (least-recently-used) 	& strikt k - kompetitiv da Markierungsalgorithmus \\ \hline
  FWF (flush-when-full)  		& strikt k - kompetitiv da Markierungsalgorithmus \\ \hline
  FIFO (first-in-first-out)  	& kein MArkierungsalgorithmus, dennoch k - kompetitiv \\ \hline
  LFU (least-frequently-used)   & nicht kompetitiv\\ \hline
  LIFO (last-in-first-out)  	& nicht kompetitiv\\ \hline
  LFD (longest-forward-distance)& optimaler Offline - Algorithmus \\ \hline
 \end{tabular}

 

\end{document}
